% template from http://rpi.edu/dept/arc/training/latex/resumes/
\documentclass{res}
%\usepackage{fullpage}
\setlength{\topmargin}{-.25in}
\setlength{\textheight}{9in} %increase text height to fit on 1-page

\usepackage{amssymb}
%\renewcommand{labelitemi}{\hfill$\square$}
\newcommand{\sqitem}{\item[\tiny$\blacksquare$]}
\newcommand{\sqlist}{\begin{list}{$\bullet$}
  { \setlength{\itemsep}{0pt}
	\setlength{\parsep}{0pt}
	\setlength{\topsep}{0pt}
	\setlength{\partopsep}{0pt}
	\setlength{\leftmargin}{6.0em}
	%\setlength{\labelwidth}{5em}
	\setlength{\labelsep}{2.5em} } }
\newcommand{\sqend}{\end{list}}


\begin{document}

\name{Eric Wustrow\\https://ericw.us/trow}
  % \\[12pt] adds a blank line after name

\address{\emph{Current Address}	\\
			1885 Fuller Rd	\\
			Ann Arbor, MI 48105			\\
										\\
			phone: 734.330.8702			\\
			email: ewust@umich.edu}
%\address{\emph{Permanent Address}	\\
%			13768 Howen Dr				\\
%			Saratoga, CA 95070}

\begin{resume}

\section{Education}
	\textbf{University of Michigan}  Expected graduation Fall 2015%& 2010-present \\
	\sqlist	
		\sqitem Ph.D. in Computer Science and Engineering 
        \sqitem Advised by J. Alex Halderman
	\sqend

	\textbf{University of Michigan}
	\sqlist
		\sqitem B.S.E. in Computer Engineering, May 2010.
	\sqend

\if0
\section{Courses}
	\begin{tabular}{l l}
	\textbf{EECS 373} Microprocessor based Systems & 
					\textbf{EECS 470} Computer Architecture \\
	\textbf{EECS 451} DSP and analysis & 
					\textbf{EECS 482} Intro to Operating Systems \\
	\textbf{EECS 461} Embedded Control Systems & 
					\textbf{EECS 483} Compiler Construction \\
											  &
					\textbf{EECS 588} Computer \& Network Security \\
	\end{tabular}
\fi

\section{Research}

I am currently a PhD candidate at the University of Michigan in Computer
Security, working with Professor J. Alex Halderman. My research focuses on
\textbf{network security}, including Internet measurement, anticensorship, and protocol
analysis.  Previously, I have worked to expose vulnerabilities in insecure
electronic voting systems in the US and abroad, found widespread weaknesses in
cryptographic protocol implementations, and have developed new tools and
techniques for circumventing large-scale Internet censorship in countries such
as Iran and China.


\section{Publications}

    \textbf{Security Analysis of a Full-Body Scanner} \\
    Keaton Mowery, Eric Wustrow, Tom Wypych, Corey Singleton, Chris Comfort, Eric Rescorla, Stephen Checkoway, J. Alex Halderman, and Hovav Shacham \\
    In \emph{Proc. 23rd USENIX Security Symposium} \\
    (\textbf{USENIX 2014}), August 2014.

    \textbf{TapDance: End-to-Middle Anticensorship without Flow Blocking} \\
    Eric Wustrow, Colleen M. Swanson, and J. Alex Halderman \\
    In \emph{Proc. 23rd USENIX Security Symposium} \\
    (\textbf{USENIX 2014}), August 2014.

    \textbf{Elliptic Curve Cryptography in Practice} \\
    Joppe W. Bos, J. Alex Halderman, Nadia Heninger, Jonathan Moore, Michael Naehrig, and Eric Wustrow \\
    In \emph{Proc. 18th International Conference on Financial Cryptography and Data Security} \\
    (\textbf{FC 2014}), March 2014.

    \textbf{ZMap: Fast Internet-wide Scanning and its Security Applications} \\
    Zakir Durumeric, Eric Wustrow, and J. Alex Halderman \\
    In \emph{Proc. 22nd USENIX Security Symposium} \\
    (\textbf{USENIX 2013}), August 2013.

    \textbf{CAge: Taming Certificate Authorities by Inferring Restricted Scopes} \\
    James Kasten, Eric Wustrow, and J. Alex Halderman \\
    In \emph{Proc. 17th International Conference on Financial Cryptography and Data Security} \\
    (\textbf{FC 2013}), April 2013.

    \textbf{Mining Your Ps and Qs: Widespread Weak Keys In Network Devices} \\
    \emph{Awarded Best Paper} \\
    Nadia Heninger, Zakir Durumeric, Eric Wustrow, and J. Alex Halderman \\
    In \emph{Proc. 21st USENIX Security Symposium} \\
    (\textbf{USENIX 2012}), August 2012.

    \textbf{Attacking the Washington, D.C. Internet Voting System} \\
    Scott Wolchok, Eric Wustrow, Dawn Isabel, and J. Alex Halderman \\
    In \emph{Proc. 16th Financial Cryptography and Data Security} \\
    (\textbf{FC 2012}), February 2012.

    \textbf{Telex: Anticensorship in the Network Infrastructure} \\
    Eric Wustrow, Scott Wolchok, Ian Goldberg and J. Alex Halderman \\
    In \emph{Proc. 20th USENIX Security Symposium} \\
    (\textbf{USENIX 2011}), August 2011.

	\textbf{Internet Background Radiation Revisited} \\
	Eric Wustrow, Manish Karir, Michael Bailey, Farnam Jahanian and Geoff Houston \\
	In \emph{Proc. 10th Internet Measurement Conference} \\
	(\textbf{IMC 2010}), November 2010.

	\textbf{Security Analysis of India's Electronic Voting  Machines} \\ 
	Scott Wolchok, Eric Wustrow, J. Alex Halderman, Hari K. Prasad, Arun Kankipati, \\
	Sai Krishna Sakhamuri, Vasavya Yagati, and Rop Gonggrijp \\
	In \emph{Proc. 17th ACM Conference on Computer and Communications 
	Security} \\
	(\textbf{CCS 2010}), October 2010.

	\textbf{PE-ARP: Port Enhanced ARP for IPv4 Address Sharing} \\
	Manish Karir, Eric Wustrow, Jim Rees \\
	Merit Networks Technical Report, July 2009.	

\section{Honors and Awards}

    \textbf{Best Paper of USENIX 2012} \\
    for ``Mining Your Ps and Qs: Detection of Widespread Weak Keys in Network Devices''

    \textbf{Runner-up for 2012 PET Award for Outstanding Research in Privacy Enhancing Technologies} \\
    for ``Telex: Anticensorship in the Network Infrastructure''

    \textbf{GSI Honorable mention} \\
    for assistance in teaching EECS 588 ``Computer and Network Security'' (2010-2011)

    \textbf{NSF Graduate Research Fellowship} \\
    for research in combating Internet censorship by state-level actors. 2011-present

    \textbf{OpenITP Fellowship at the New American Foundation} \\
    for research in the area of Internet Freedom and anticensorship. 2013-2014

\section{Talks}

    \textbf{Security Analysis of a Full-Body Scanner} \\
    31st Chaos Communication Congress (Hamburg), December 2014

    \textbf{Anticensorship in the Network Infrastructure} \\
    RIPE 68 (Warsaw), May 2014

    \textbf{Elliptic Curve Cryptography in Practice} \\
    Financial Cryptography 2014 (Barbados), March 2014

    \textbf{Finding whom to blame: Network Tools} \\
    Michigan Hackers Tech Talk (Ann Arbor), October 2012

    \textbf{Telex: Anticensorship in the Network Infrastructure} \\
    Boston Freedom in Online Communication, March 2013
    RightsCon Circumvention Tech Summit (Rio de Janeiro), May 2012 \\
    NANOG 54 (San Diego), February 2012 \\
    USENIX Security Symposium (San Francisco), August 2011

    \textbf{Internet Background Radiation Revisited} \\
    Internet Measurement Conference (Melbourne, Australia), November 2010

    \textbf{Security Analysis of India's Electronic Voting Machines} \\
    ACM Conference on Computer and Communications Security (Chicago), October 2010

    \textbf{BGPBotz: IM-based route views} \\
    NANOG 47 (Dearborn, Michigan), October 2009

\section{Teaching}
    \textbf{EECS 388} Introduction to Computer Security (Winter 2015)
        \sqlist
        \sqitem Co-instructed two sections of undergraduate security course with Professor Z. Morley Mao
        \sqend

    \textbf{EECS 588} Computer and Network Security (Winter 2011-2014)
        \sqlist
        \sqitem Guest taught several lectures and advised students on course projects
        \sqend

    \textbf{EECS 398} Introduction to Computer Security (Fall 2011, 2013)
        \sqlist
        \sqitem Guest taught several lectures
        \sqend

    \textbf{Camp CAEN} Computer programming summer camp (Summer 2008)
        \sqlist
        \sqitem Taught middle- and high-school students basic HTML, CSS, Javascript, and Java.
        \sqend

\section{Professional Service}
    \sqlist
    \sqitem 2013 Program Committee member for USENIX Workshop on Free and Open Communications on the Internet (FOCI '13)
    \sqitem 2012 Reviewer for IEEE/ACM Transactions on Networking
    \sqend


\end{resume}

\end{document}
