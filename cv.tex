% https://ewust.pad.jhalderm.com/build/resume.pdf

% template from http://rpi.edu/dept/arc/training/latex/resumes/
\documentclass[margin,11pt]{res} %ethertex: https://raw.githubusercontent.com/ewust/resume/master/res.cls
\usepackage{fullpage}
\usepackage{fancyhdr}
\usepackage{color}
\definecolor{MyDarkBlue}{rgb}{0.05,0,0.5} 
\usepackage[
  letterpaper=true,breaklinks=true,colorlinks=true,linkcolor=black,%
citecolor=black,urlcolor=MyDarkBlue,bookmarks=false,bookmarksopen=false,%
  pdftex]{hyperref}

\textwidth=5.2in
\setlength{\topmargin}{-.125in}
\setlength{\textheight}{9in} %increase text height to fit on 1-page
\setlength{\parskip}{0.8em}

\usepackage[T1]{fontenc}
\usepackage{kpfonts}
\usepackage{changepage}

\usepackage{amssymb}
%\renewcommand{labelitemi}{\hfill$\square$}
\newcommand{\sqitem}{\item[\tiny$\blacksquare$]}
\newcommand{\sqlist}{\begin{list}{$\bullet$}
  { \setlength{\itemsep}{0pt}
        \setlength{\parsep}{0pt}
        \setlength{\topsep}{0pt}
        \setlength{\partopsep}{0pt}
        \setlength{\leftmargin}{6.0em}
        %\setlength{\labelwidth}{5em}
        \setlength{\labelsep}{2.5em} } }
\newcommand{\sqend}{\end{list}}
\newcommand{\inwidth}{1in}
\renewcommand{\medskip}{}

\pagestyle{fancy}
\fancyhead{}
\fancyfoot{}
\fancyfoot[R]{\thepage}
\renewcommand{\headrulewidth}{0pt}

\begin{document}

\begin{adjustwidth}{-1.3in}{0pt}
\noindent{\LARGE\bf Eric Wustrow}
\vspace{20pt}

\noindent 
\parbox[m]{4in}{
Assistant Professor \\
University of Colorado Boulder\\
Electrical, Computer, and Energy Engineering\\
425 UCB $\cdot$ Boulder, CO \,80309
}
\begin{tabular}{ll}
{Web}:&\url{https://ericw.us/trow}\\
{Email}:&ewust@colorado.edu\\
{Phone}:& 734.330.8702
\end{tabular}
\vspace{10pt}
\end{adjustwidth}

\section{\large Education}

        \emph{University of Michigan}\\
        Ph.D. in Computer Science, November 2015\\
        \emph{Advisor:} J. Alex Halderman

        \emph{University of Michigan}\\
        B.S.E. in Computer Engineering, May 2010

\vspace{6pt}
\section{\large Research}

My research focuses on \textbf{computer security} from a systems perspective. My work has spanned censorship resistance, Internet protocol security, and architecture security.

I have exposed vulnerabilities in insecure electronic voting systems in the U.S. and abroad, developed detection techniques that uncovered weaknesses in widespread cryptographic protocol implementations, and created systems for circumventing large-scale Internet censorship in countries such as Iran and China.

\vspace{6pt}
\section{\large Publications}

    \textbf{Conjure: Summoning Proxies from Unused Address Space} \\
    Sergey Frolov, Jack Wampler, Sze Chuen Tan, J. Alex Halderman, Nikita Borisov, and Eric Wustrow \\
    (to appear) In \emph{Proc.\ 26th ACM Conference on Computer and Communications Security} \\
    (\textbf{CCS 2019}), November 2019. \\
    Acceptance rate: 16\% (117/722)

    \textbf{This is Your President Speaking: Spoofing Alerts in 4G LTE Networks} \\
    Gyuhong Lee, Jihoon Lee, Jinsung Lee, Youngbin Im, Max Hollingsworth, Eric Wustrow, Dirk Grunwald, and Sangtae Ha \\
    In \emph{Proc. of the 17th ACM International Conference on Mobile Systems, Applications and Services} \\
    (\textbf{MobiSys 2019}), July 2019. \\
    \textbf{$\star$ Awarded Best Paper.} \\
    Acceptance rate: 23\% (40/172)

    \textbf{ExSpectre: Hiding Malware in Speculative Execution} \\
    Jack Wampler, Ian Martiny, and Eric Wustrow \\
    In \emph{Proc. of Network and Distributed System Security Symposium} \\
    (\textbf{NDSS 2019}), February 2019. \\
    Acceptance rate: 17\% (89/521)

    \textbf{\href{https://tlsfingerprint.io}{The use of TLS in Censorship Circumvention}} \\
    Sergey Frolov and Eric Wustrow \\
    In \emph{Proc. of Network and Distributed System Security Symposium} \\
    (\textbf{NDSS 2019}), February 2019. \\
    Acceptance rate: 17\% (89/521)


    \textbf{Breaking the Trust Dependence on Third Party Processes for Reconfigurable Secure Hardware} \\
    Aimee Coughlin, Greg Cusack, Jack Wampler, Eric Keller, and Eric Wustrow \\
    In \emph{Proc.\ 27th ACM/SIGDA International Symposium on Field-Programmable Gate Arrays}\\
    (\textbf{FPGA 2019}), February 2019. \\
    % Acceptance rate ??

    \textbf{The Proof is in the Pudding - Proofs of Work for Solving Discrete Logarithms} (Short paper) \\
    Marcella Hastings, Nadia Heninger, and Eric Wustrow \\
    In \emph{Proc.\ 23rd Intl.\ Conference on Financial Cryptography and Data Security} \\
    (\textbf{FC 2019}), February 2019.\\
    Acceptance rate: 22\% (40/178)   % 33 + 7 short

    \textbf{Proof-of-Censorship: Enabling Centralized Censorship-resistant Content Providers} \\
    Ian Martiny, Ian Miers, and Eric Wustrow \\
    In \emph{Proc.\ 22nd Intl.\ Conference on Financial Cryptography and Data Security} \\
    (\textbf{FC 2018}), February 2018. \\
    Acceptance rate: 26\% (29/109)

    \textbf{Initial Measurements of the Cuban Street Network} (Short paper) \\
    Eduardo E P Pujol, Will Scott, Eric Wustrow, J Alex Halderman \\
    In \emph{Proc. of the 2017 Internet Measurement Conference} \\
    (\textbf{IMC 2017}), November 2017. \\
    Acceptance rate: 23\% (42/179)

    \textbf{An ISP-scale deployment of TapDance} \\
    Sergey Frolov, Fred Douglas, Will Scott, Allison McDonald, Benjamin VanderSloot, Rod Hynes, Adam Kruger, Michalis Kallitsis, David G Robinson, Steve Schultze, Nikita Borisov, Alex Halderman, and Eric Wustrow \\
    In \emph{Proc. of USENIX Workshop on Free and Open Communications on the Internet} \\
    (\textbf{FOCI 2017}), August 2017. \\
    Acceptance rate: 55\% (10/18)

%\newpage
    \textbf{Trusted click: Overcoming security issues of NFV in the cloud} \\
    Michael Coughlin, Eric Keller, and Eric Wustrow \\
    In \emph{Proc. of the ACM International Workshop on Security in Software Defined Networks \& Network Function Virtualization} \\
    (\textbf{SDN-NFVSec}), March 2017. \\
    \textbf{$\star$ Awarded Best Paper.}

\newpage
    \textbf{\href{https://github.com/ewust/DDoSCoin}{DDoSCoin: Cryptocurrency with a Malicious Proof-of-Work}} \\
    Eric Wustrow, and Benjamin VanderSloot \\
    In \emph{Proc. of the 10th USENIX Workshop on Offensive Technologies} \\
    (\textbf{WOOT 2016}), August 2016. \\
    Acceptance rate: 47\% (21/44)

    \textbf{\href{https://weakdh.org/}{Imperfect Forward Secrecy: How Diffie-Hellman Fails in Practice}} \\
    David Adrian, Karthikeyan Bhargavan, Zakir Durumeric, Pierrick Gaudry, Matthew Green,
    J. Alex Halderman, Nadia Heninger, Drew Springall, Emmanuel Thom\'e, Luke Valenta,
    Benjamin VanderSloot, Eric Wustrow, Santiago Zanella-B\'eguelink, and Paul Zimmermann \\
    In \emph{Proc.\ 22nd ACM Conference on Computer and Communications Security} \\
    (\textbf{CCS 2015}), October 2015. \\
    \textbf{$\star$ Awarded Best Paper.} \\
    Acceptance rate: 20\% (128/646)

    \textbf{\href{https://keysforge.com/}{Replication Prohibited: Attacking Restricted Keyways with 3D Printing}} \\
    Ben Burgess, Eric Wustrow, and J. Alex Halderman \\
    In \emph{Proc. of the 9th USENIX Workshop on Offensive Technologies} \\
    (\textbf{WOOT 2015}), August 2015. \\
    Acceptance rate: 35\% (20/57)

    \textbf{\href{https://www.radsec.org/paper.html}{Security Analysis of a Full-Body Scanner}} \\
    Keaton Mowery, Eric Wustrow, Tom Wypych, Corey Singleton, Chris Comfort, Eric Rescorla, Stephen Checkoway, J. Alex Halderman, and Hovav Shacham \\
    In \emph{Proc.\ 23rd USENIX Security Symposium} \\
    (\textbf{USENIX Security 2014}), August 2014. \\
    Acceptance rate: 19\% (67/350)

    \textbf{\href{https://jhalderm.com/pub/papers/tapdance-sec14.pdf}{TapDance: End-to-Middle Anticensorship without Flow Blocking}} \\
    Eric Wustrow, Colleen M. Swanson, and J. Alex Halderman \\
    In \emph{Proc.\ 23rd USENIX Security Symposium} \\
    (\textbf{USENIX Security 2014}), August 2014. \\
    Acceptance rate: 19\% (67/350)

    \textbf{\href{http://eprint.iacr.org/2013/734}{Elliptic Curve Cryptography in Practice}} \\
    Joppe W. Bos, J. Alex Halderman, Nadia Heninger, Jonathan Moore,\\ Michael Naehrig, and Eric Wustrow \\
    In \emph{Proc.\ 18th Intl.\ Conference on Financial Cryptography and Data Security} \\
    (\textbf{FC 2014}), March 2014. \\
    Acceptance rate: 22\% (31/138)

    \textbf{\href{https://zmap.io/paper.html}{ZMap: Fast Internet-wide Scanning and its Security Applications}} \\
    Zakir Durumeric, Eric Wustrow, and J. Alex Halderman \\
    In \emph{Proc.\ 22nd USENIX Security Symposium} \\
    (\textbf{USENIX Security 2013}), August 2013. \\
    Acceptance rate: 16\% (45/277)


\newpage
    \textbf{\href{https://jhalderm.com/pub/papers/cage-fc13.pdf}{CAge: Taming Certificate Authorities by Inferring Restricted Scopes}} \\
    James Kasten, Eric Wustrow, and J. Alex Halderman \\
    In \emph{Proc.\ 17th Intl.\ Conference on Financial Cryptography and Data Security} \\
    (\textbf{FC 2013}), April 2013.

    \textbf{\href{https://factorable.net/weakkeys12.conference.pdf}{Mining Your Ps and Qs: Widespread Weak Keys In Network Devices}} \\
    Nadia Heninger, Zakir Durumeric, Eric Wustrow, and J. Alex Halderman \\
    In \emph{Proc. 21st USENIX Security Symposium} \\
    (\textbf{USENIX Security 2012}), August 2012. \\
    \textbf{$\star$ Awarded Best Paper.} \\
    Acceptance rate: 19\% (43/222)

    \textbf{\href{https://ericw.us/trow/dc-internet-voting-fc.pdf}{Attacking the Washington, D.C. Internet Voting System}} \\
    Scott Wolchok, Eric Wustrow, Dawn Isabel, and J. Alex Halderman \\
    In \emph{Proc. 16th Financial Cryptography and Data Security} \\
    (\textbf{FC 2012}), February 2012. \\
    Acceptance rate: 26\% (33/88)

    \textbf{\href{https://telex.cc/pub/telex-usenixsec11.pdf}{Telex: Anticensorship in the Network Infrastructure}} \\
    Eric Wustrow, Scott Wolchok, Ian Goldberg and J. Alex Halderman \\
    In \emph{Proc.\ 20th USENIX Security Symposium} \\
    (\textbf{USENIX Security 2011}), August 2011. \\
        \textbf{$\star$ PET Award Runner-up.} \\
    Acceptance rate: 17\% (35/204)

    \textbf{\href{https://ericw.us/trow/imc10-wustrow.pdf}{Internet Background Radiation Revisited}} \\
        Eric Wustrow, Manish Karir, Michael Bailey, Farnam Jahanian and\\ Geoff Houston \\
        In \emph{Proc.\ 10th Internet Measurement Conference} \\
        (\textbf{IMC 2010}), November 2010.\\
        Acceptance rate: 22\% (47/211)

    \textbf{\href{http://www.cse.umich.edu/~jhalderm/pub/papers/evm-ccs10.pdf}{Security Analysis of India's Electronic Voting  Machines}} \\ 
        Scott Wolchok, Eric Wustrow, J. Alex Halderman, Hari K. Prasad, Arun Kankipati, \\
        Sai Krishna Sakhamuri, Vasavya Yagati, and Rop Gonggrijp \\
        In \emph{Proc.\ 17th ACM Conference on Computer and Communications Security} \\
        (\textbf{CCS 2010}), October 2010. \\
        \textbf{$\star$ Highest Rated Submission.} \\
        Acceptance rate: 17\% (55/320)

    \textbf{\href{https://ericw.us/trow/pe-arp.pdf}{PE-ARP: Port Enhanced ARP for IPv4 Address Sharing}} \\
        Manish Karir, Eric Wustrow, Jim Rees \\
        Merit Networks Technical Report, July 2009.

\if0
\vspace{6pt}
\section{\large Bibliometrics}

Citations: 584\\
$h$-index: 10\\
\url{https://scholar.google.com/citations?user=becbtxgAAAAJ}
\fi

%\newpage


\vspace{6pt}
\section{\large Broader\\Impact}

\textbf{\href{https://tlsfingerprint.io}{TLS fingerprints in Censorship Circumvention}} (2018) \\
In this ongoing study, we collect real-world Internet traffic and compare the TLS ``fingerprints'' of censorship circumvention tools to real-world implementations. This is useful for finding and fixing tools at risk of being blocked by censors, and helped guide design of our purpose-built library (uTLS) for mimicking TLS implementations.

\textbf{\href{https://refraction.network}{Refraction Networking}} (2018-) \\
We have deployed a fundamentally new form of censorship circumvention tool that places proxies in the middle of the network, at Internet service providers (ISPs) outside censoring countries. We partnered with university and research ISPs, as well as a popular censorship circumvention tool, and are currently providing ongoing Internet access to tens of thousands of users in censored regions.

\textbf{\href{https://radsec.org}{Analysis of a Full Body Scanner}} (2014)\\
We revealed that X-ray backscatter full-body scanners previously used in airports were insufficient to detect the non-metallic threats they were specifically intended to find. This work raised serious questions about TSA's procedures for purchasing and deploying security technologies.

\textbf{\href{https://zmap.io/}{ZMap Internet-Wide Scanner}} (2013)\\
ZMap is an open-source, Internet-wide network scanner tool that is able to probe the entire IPv4 address space in under 45 minutes, over 1000 times faster than previous approaches. Now a major open-source project, it has been adopted widely by researchers performing Internet security measurement.

\textbf{\href{https://factorable.net/}{Detection of Widespread Weak Keys in Network Devices}} (2012)\\
By scanning the Internet for TLS and SSH hosts, we discovered that millions of embedded networked devices had generated weak cryptographic keys using insufficient entropy sources. We disclosed vulnerabilities to more than 60 network device makers and spawned major changes to the Linux kernel's random number generator.

\textbf{\href{https://telex.cc/}{Telex Anticensorship System}} (2011)\\
Telex is a fundamentally new form of censorship circumvention that places proxies in the middle of the network, at Internet service providers (ISPs) outside censoring countries.  This makes them difficult for censors to block without blocking large amounts of unrelated traffic. I'm now working with a large ISP to deploy a Telex testbed.\looseness=-1

\textbf{\href{http://indiaevm.org/}{Analysis of India's E-Voting System}} (2010)\\
We demonstrated low-tech attacks that could compromise India's nation-wide electronic voting machines, showing that the system was not tamperproof as the government claimed.  As a result, India is working to deploy new machines that add a paper audit trail, changing how the country votes.

\section{\large Funding}

\textbf{SDR LTE Network Testbed and RESPONS} \\
Source: Public Saftey Communications Research (PSCR) Division - NIST \\
Award Amount: \$1,502,796 (co-PI) \\
06/01/2017 - 05/31/2020

\textbf{Decoy Routing: Internet Freedom in the Network's Core} \\
Source: United States Department of State \\
Award Amount: \$485,696 (co-PI; total award: \$4,000,026) \\
04/01/2016 - 09/31/2019



\vspace{6pt}
%\newpage
\section{\large Honors and Awards}
    \textbf{Best Paper of Mobisys 2019}
    for ``This is Your President Speaking: Spoofing Alerts in 4G LTE Networks.''

    \textbf{Best Paper of CCS 2015}
    for ``Imperfect Forward Secrecy: How Diffie-Hellman Fails in Practice.''

    \textbf{Best Paper of USENIX Security 2012}
    for ``Mining Your Ps and Qs: Detection of Widespread Weak Keys in Network Devices.''

    \textbf{Runner-up for 2012 PET Award} for Outstanding Research in Privacy Enhancing Technologies for ``Telex: Anticensorship in the Network Infrastructure.''

 %   \textbf{GSI Honorable mention}
%    for assistance in teaching EECS 588 ``Computer and Network Security'' (2010--11).

    \textbf{NSF Graduate Research Fellowship}
    for research in combating Internet censorship by state-level actors (2011-2015).
    
    \textbf{OpenITP Fellowship at the New American Foundation}
    for research in the area of Internet Freedom and anticensorship (2013--14).

\vspace{6pt}
\section{\large Invited Talks and Panels}

\emph{Refraction Networking: Deploying next-generation censorship circumvention} \\
    RightsCon (Tunis), June 2019

\emph{Panel: Technologist Reactions and Perspectives} \\
    The Legal Pitfalls of Ethical Hacking - Silicon Flatirons (Denver), December 2018

\emph{Understanding Cryptocurrencies} \\
    Association of Certified Anti-Money Laundering Specialists (ACAMS) Colorado Chapter (Denver), June 2018

\emph{Deploying Anticensorship in the Network} \\
    CyLab (CMU, Pittsburgh), March 2018 \\
    SUMIT 2017 (Ann Arbor), October 2017

\emph{Panel: Censorship Circumvention, From Academia to Practice} \\
    Annual Computer Security Applications Conference (ACSAC - Los Angeles), December 2016

\emph{Replication Prohibited: 3D Printed key attacks} \\
    32nd Chaos Communication Congress (Hamburg), December 2015

\emph{Security Analysis of a Full-Body Scanner} \\
    31st Chaos Communication Congress (Hamburg), December 2014

\emph{Anticensorship in the Network Infrastructure} \\
    RIPE 68 (Warsaw), May 2014

%\emph{Elliptic Curve Cryptography in Practice} \\
%    Financial Cryptography 2014 (Barbados), March 2014

\emph{Finding Whom to Blame: Network Tools} \\
    Michigan Hackers Tech Talk (Ann Arbor), October 2012

\emph{Telex: Anticensorshp in the Network Infrastructure} \\
    Boston Freedom in Online Communication, March 2013\\
    RightsCon Circumvention Tech Summit (Rio de Janeiro), May 2012 \\
    NANOG 54 (San Diego), February 2012 %\\
%    USENIX Security Symposium (San Francisco), August 2011

%\emph{Internet Background Radiation Revisited} \\
%    Internet Measurement Conference (Melbourne, Australia), November 2010

%\emph{Security Analysis of India's Electronic Voting Machines} \\
%    ACM CCS (Chicago), October 2010

%\emph{BGPBotz: IM-based route views} \\
%    NANOG 47 (Dearborn, Michigan), October 2009

\vspace{6pt}
\section{\large Professional Service}

    \emph{Department:}\\
        Chair Search Committee 2018-2019 \\
        Graduate Committee 2018-2019 \\
        Big Data / Machine Learning / Security Faculty Search Committee 2017-2018 \\
        Computer Science Faculty Search Committee 2016-2017 \\
        Computer Engineering Faculty Search Committee 2016-2017 \\
        Graduate Committee 2015-2016 \\

     \emph{Program committee member:}\\
            Internet Measurement Conference (IMC) 2019  \\
            USENIX Security Symposium 2018, 2019    \\
            Financial Cryptography (FC) 2017, 2018, 2019 \\
            TheWebConf (formerly WWW) 2017, 2019 \\
            Research in Attacks, Intrusions and Defenses (RAID) 2018 \\
            USENIX Workshop on Free and Open Communications on the Internet (FOCI) 2013, 2016, 2018. \\
     
    \emph{External reviewer:}
    USENIX Security Symposium 2014,
    ACM Conference on Computer and Communications Security (CCS) 2012--15,
    ACM Internet Measurement Conference (IMC) 2015,
    IEEE/ACM Transactions on Networking 2012.

\vspace{6pt}
\section{\large Teaching}
    \emph{ECEN 5032: Cryptocurrency Security} (Spring 2019)

    \emph{ECEN 3350: Programming Digital Systems} (Fall 2018, Fall 2019)

    \emph{ECEN 5003: Censorship Circumvention} (Fall 2017)

    \emph{ECEN 5032: Introduction to Computer Security} (Fall 2016, Spring 2017, Spring 2018)

    \emph{ECEN 5014: Computer Security and Privacy} (Spring 2016)

    \emph{EECS 388: Introduction to Computer Security}  (Spring 2015)\\
    \textbf{Co-instructor} of undergraduate security course, with Professor Z. Morley Mao. Responsible for teaching half of lecture sections and developing curriculum.

    \emph{TA EECS 588: Computer and Network Security} (2011--2014)\\
    Served as teaching assistant, guest lectured, and advised course project groups.

    %\emph{EECS 398: Introduction to Computer Security}  (2011, 2013)\\
    %Guest taught several lectures each year.

    \emph{Camp CAEN: Computer programming summer camp}  (2008)\\
    Taught middle- and high-school students HTML, CSS, Javascript, and Java.


\if0
\vspace{6pt}
\section{\large Internships}

    \emph{Square, Inc.} --- Software Security Intern (2014) \\
    Implemented new secure channel for next-generation credit card reader.
    
    \emph{Qualcomm} --- Software Intern (2010) \\
    Worked with CDMA Technologies team testing LTE software functionality.
    
    \emph{Merit Networks} --- Network Engineering Intern (2009) \\
    Modified Linux kernel to support new IPv4 sharing concepts.
   % Wrote a BGP looking-glass chat bot and presented tool at NANOG 47.

    \emph{Radio Aurora Explorer (RAX) Satellite} (2009) \\
    Designed hardware and software for orbital experiments.
\fi

%\newpage

%\vspace{6pt}

\if0
\section{\large References}
    
\textbf{J. Alex Halderman}\\
\emph{Assistant Professor}\smallskip\\
Computer Science \& Engineering\\
University of Michigan\\
2260 Hayward Street\\
Ann Arbor, MI \,48109\\
(734) 647-1806\\
jhalderm@eecs.umich.edu
\vspace{6pt}

\textbf{Michael Bailey}\\
    \emph{Associate Professor}\smallskip\\
    {ECE Department\\
    University of Illinois\\
    1308 W Main Street \\
    Urbana, IL \,61801-2307\\
     (734) 320-3810\\
    mdbailey@illinois.edu}
\vspace{6pt}

\textbf{Z. Morley Mao}\\
\emph{Associate Professor}\smallskip\\
Computer Science \& Engineering\\
University of Michigan\\
2260 Hayward Street\\
Ann Arbor, MI \,48109\\
(734) 763-5407\\
zmao@umich.edu
\vspace{6pt}

\textbf{Brian Noble}\\
\emph{Professor}\smallskip\\
Computer Science \& Engineering\\
University of Michigan\\
2260 Hayward Street\\
Ann Arbor, MI \,48109\\
(734) 936-2971\\
bnoble@umich.edu
\vspace{6pt}

\fi



\end{document}

